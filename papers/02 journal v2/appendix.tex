\appendix[Supplementary Material ]
\subsection{Algorithm 1: Derivation and Convergence Analysis}
\label{app:CA}
The derivation and convergence analysis of Algorithm 1 relies (for the most part) on the \textit{customized preconditioned proximal-point (cPPP)} algorithm for generalized aggregative games proposed in \cite[Algorithm~6]{belgioioso2020semi}. The objective of this appendix is to show that the proposed market-clearing game \eqref{eq:locOpt}, with cost functions and constraints sets defined in \eqref{eq:cost_def}-\eqref{eq:Cconstr_def}, satisfies all the technical conditions in \cite[Theorem~2]{belgioioso2020semi}, among which is the existence of a variational GNE, i.e., item (i) of Proposition \ref{prp:conv}. Therefore, we invoke \cite[Theorem~2]{belgioioso2020semi} to prove convergence of Algorithm \ref{alg:alg1}, i.e., item (ii) of Proposition \ref{prp:conv}. For a complete convergence analysis of the cPPP algorithm for aggregative games we refer to \cite[Appendix~C]{belgioioso2020semi}.

\smallskip
\subsubsection*{Aggregative cost functions} First, we show that the cost functions \eqref{eq:cost_def} can be cast as in\cite[Eqn.~(30)]{belgioioso2020semi}, i.e.
\begin{equation}
	\label{eq:cFcPPP}
	J_i(u_i, u_{-i}) = g_i(u_i) + {(C \avg( u))}^\top u_i,
\end{equation}
where $\avg( u) := \frac{1}{N}\sum_{i \in \mc N} u_i$ denotes the average strategy.
Let $\mc N_i = \mc N$, for all $i \in \mc N$, without loss of generality\footnote{For example, by defining, for all $i \in \mc N$, the ``dummy variables" $\{p^{\text{tr}}_{(i,j)}\}_{j \in \mc N \setminus {\mc N_i}}$ for all the prosumers that do not trade with $i$.}. In this case, $u_i \in \R^{(3+N)H}$, for all $i \in \mc N$. Moreover, let $\Xi^{\text{mg}} \in \R^{H \times (3+N)H}$ denote the matrix that selects the $p_{i}^{\text{mg}}$-component from the decision vectors $u_i$'s, and define the matrix $D:= N \diag(d^{\text{mg}}_1, \ldots, d^{\text{mg}}_H)$, where $d^{\text{mg}}_h$ is the price coefficient for the main grid power. Then, the cost functions in \eqref{eq:cost_def} can be recast as \cite[Eqn.~(30)]{belgioioso2020semi}, or \eqref{eq:cFcPPP}, with
\begin{subequations}
	\label{eq:pieces_cPPP}
	\begin{align}
		g_i(u_i) &= \nonumber
		f_{i}^{\mathrm{di}}(p_{i}^{\mathrm{di}}) + f_{i}^{\mathrm{st}}(p_{i}^{\mathrm{st}})+ f_{i}^{\mathrm{tr}} \left( \{ p_{(i,j)}^{\mathrm{tr}} \}_{j \in \mc N_i} \right),\\[-.2em]
		\label{eq:gi_cPPP}
		&
		\textstyle
		\quad +\frac{1}{N}(Db)^\top p_i^{\text{mg}},\\[.2em]
		%
		C &= (\Xi^{\text{mg}})^\top D \, \Xi^{\text{mg}}.
		\label{eq:C_cPPP}
	\end{align}
\end{subequations}


\subsubsection*{Technical assumptions} Next, we show that all the assumptions in \cite[Theorem~2]{belgioioso2020semi} are satisfied.
\begin{enumerate}[(i)]
	\item For all $i \in \mc N^+$, the cost function $J_i(u_i, u_{-i})$ in \eqref{eq:gi_cPPP} is convex in $u_i$, since all the components of $g_i$ are convex. Hence, \cite[Assumption~1]{belgioioso2020semi} holds.
	%
	\item For all $i \in \mc N^+$, the local set $\mc U_i$ in \eqref{eq:constr_def} is nonempty, closed and convex. Moreover, Slater's constraint qualification on the global feasible set $\left( \prod_{i \in \mc N^+} \mc U_i \right) \cap \mc C$ holds under an appropriate choice of the parameters. Therefore, \cite[Assumption~2]{belgioioso2020semi} is satisfied.
	%
	\item The \textit{pseudo-subdifferential} mapping of the game \eqref{eq:locOpt} reads as $F: u \mapsto \prod_{i \in \mc N}\left(\partial_{u_i} J_i(u_i, u_{-i})\right)\times \0$, since $J_{N+1}=0$. It follows by \cite[Corollary~1]{belgioioso2017convexity}, that the first term of $F$, i.e., $u \mapsto \prod_{i \in \mc N}\partial_{u_i} J_i(u_i, u_{-i})$, is maximally monotone, since  $C$ in \eqref{eq:C_cPPP} is positive semidefinite, i.e., $C=(\Xi^{\text{mg}})^\top D \, \Xi^{\text{mg}} \succeq 0$. Moreover, also the second term of $F$, i.e., the zero mapping $\0$, is maximally monotone. Therefore, it follows by \cite[Proposition~20.23]{bauschke2011convex} that their cartesian product $\prod_{i \in \mc N}\left(\partial_{u_i} J_i(u_i, u_{-i})\right)\times \0 = F$ is maximally monotone. Hence, \cite[Assumption~6]{belgioioso2020semi} holds.
	%
	\item By \cite[Lemma~1~(i)]{belgioioso2020semi}, there exists a variational GNE of the game in \eqref{eq:locOpt}, since the constraint sets $\mc U_i$ in \eqref{eq:constr_def} are bounded, and the pseudo-subdifferential mapping $F$ is monotone. Hence, \cite[Assumption~4]{belgioioso2020semi} is satisfied.
\end{enumerate} 
{\hfill $\blacksquare$}
%\subsection*{Step-size selection}
%\blue{to complete}.


\subsection{Alternating Projection for Operational Feasibility}
\label{APA}
In this appendix, we propose an efficient algorithm to compute the projection onto the set $\mathcal{U}_{N+1}$  (line 30 in Algorithm~\ref{alg:alg1}). First, let us recall the structure of $u_{N+1}$, i.e.,
$$u_{N+1}=\col\left(\{\theta_y, v_y, p_y^{\mathrm{tg}},\{p_{(y,z)}^{\ell},q_{(y,z)}^{\ell}\}_{z \in \mc B_y} \}_{y\in\mc B} \right),$$ and let us define the sets
\begin{align}
	\mc S_1 &:= \{u_{N+1} \mid \text{\eqref{eq:line} and \eqref{eq:grid_ex} hold}\},\\
	%
	\mc S_2 &:= \{u_{N+1} \mid \text{\eqref{eq:pf_p} and \eqref{eq:pf_q} hold}\},
\end{align}
such that $\mathcal{U}_{N+1} = \mc S_1 \cap \mc S_2$. 
%
The proposed method, summarized in Algorithm~\ref{alg:alg2}, is essentially a \textit{Douglas--Rachford splitting} (DRS) \cite[\S~26.3]{bauschke2011convex} applied to the \textit{best approximation problem} $ {\argmin}_{\xi \in \mc S_1 \cap \mc S_2}  \|\xi-u_{N+1}\| =\proj_{\mc S_1 \cap \mc S_2}(u_{N+1}) $, see e.g. \cite[\S~4.3]{bauschke2015projection} for a formal derivation of the algorithm.

%
Unlike $\mc U_{N+1}$, the projections onto $\mc S_1$ and $\mc S_2$ have closed-form expressions, hence Algorithm \ref{alg:alg2} only involves elementary operations. Specifically,
%\begin{align*}
$	\proj_{\mc S_1}(u_{N+1})=
u_{N+1}^+$, where
\begin{align*}
	\theta_y^+ &=
	\begin{cases}
		\underline{\theta}_y, & \text{if } \theta_y < \underline{\theta}_y\\
		\overline{\theta}_y, & \text{if } \theta_y > \overline{\theta}_y\\
		\theta_y, & \text{otherwise } \
	\end{cases}, \quad \
	%
	v_y^+ = 
	\begin{cases}
		\underline{v}_y, & \text{if } v_y < \underline{v}_y\\
		\overline{v}_y, & \text{if } v_y > \overline{v}_y\\
		v_y, & \text{otherwise } \
	\end{cases},
	\\
	%
	{p_y^{\text{tg}}}^+ &= 
	\begin{cases}
		p_y^{\text{tg}}, & \text{if } y \in \mc B^{\text{mg}}\\
		0, & \text{otherwise} 
	\end{cases},
\end{align*}
and for all $y \in \mc B$, $z \in \mc B_z$, and  $h \in \mc H$
\begin{align*}
	\begin{array}{l}
		L_{(y,z),h} = \max \left\{ \|\col(p^\ell_{(y,z),h} , q^\ell_{(y,z),h} \|, \, \overline{s}_{(y,z)} \right\},\\[.2em]
		%
		{(p^\ell_{(y,z),h})}^+  = \tfrac{\overline{s}_{(y,z)}}{
			L_{(y,z),h}
		}\,  p^\ell_{(y,z),h}, \\[.2em]
		%
		{(q^\ell_{(y,z),h})}^+  = \tfrac{\overline{s}_{(y,z)}}{
			L_{(y,z),h}
		}\, q^\ell_{(y,z),h}.
	\end{array}
\end{align*}
Whereas, since $\mc S_2$ is an affine set, a closed-form expression for $\proj_{\mc S_2}$ is given in \cite[Example 29.17(ii)]{bauschke2011convex}.
%The projection onto $C_2$ can be computed by solving a quadratic programming (e.g. via lsqlin, quadprog, osqp, etc...) with appropriate matrices.

\begin{algorithm}[]
	\caption{DRS for computing $\proj_{\mc U_{N+1}}(u_{N+1})$}
	\label{alg:alg2}
	\begin{algorithmic}[1]
		
		\smallskip
		\State Initialize $\xi(0) \in \bb R^{n_{N+1}}$, and set $\eta \in (0,2)$
		\IUC{ }
		
		\smallskip
		\State
		$z(k) = \proj_{\mc S_1}(\frac{1}{2} \xi(k) + \frac{1}{2} u_{N+1})$ 
		
		
		\smallskip
		\State
		$\xi(k+1) = \xi(k) + \eta \left( \proj_{\mc S_2}    (2z(k)-\xi(k)) - z(k)
		\right)$
		\EndIUC
		
	\end{algorithmic}
\end{algorithm}